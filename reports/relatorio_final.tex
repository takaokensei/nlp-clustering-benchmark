\documentclass[12pt,a4paper,twoside]{article}

%----------------------------------------------------------------------------------------
% ENCODING AND LANGUAGE
%----------------------------------------------------------------------------------------
\usepackage[utf8]{inputenc}
\usepackage[T1]{fontenc}
\usepackage[brazilian]{babel}
\usepackage[tracking=true]{microtype}

%----------------------------------------------------------------------------------------
% FONTS
%----------------------------------------------------------------------------------------
% Serif (Body): Cormorant Garamond (fallback to EB Garamond)
\IfFileExists{CormorantGaramond.sty}
    {\usepackage{CormorantGaramond}}
    {\usepackage{ebgaramond}}

% Sans-Serif (Headings): Lato
\usepackage[defaultsans]{lato}

% Monospace (Code): Fira Mono
\usepackage{FiraMono}

%----------------------------------------------------------------------------------------
% MATHEMATICS
%----------------------------------------------------------------------------------------
\usepackage{amsmath}
\usepackage{amsthm}
\usepackage{mathtools}
\usepackage{bm}
\usepackage{amssymb}
\usepackage{siunitx}

%----------------------------------------------------------------------------------------
% LAYOUT AND GEOMETRY
%----------------------------------------------------------------------------------------
\usepackage[a4paper,
    left=3cm, right=2cm,
    top=3cm, bottom=2cm,
    headheight=22pt, headsep=20pt,
    footskip=40pt]{geometry}
\usepackage{fancyhdr}
\usepackage{lastpage}
\usepackage{setspace}
\usepackage{lipsum} 

%----------------------------------------------------------------------------------------
% GRAPHICS AND FIGURES
%----------------------------------------------------------------------------------------
\usepackage{graphicx}
\usepackage{xcolor}
\usepackage{tikz}
\usetikzlibrary{positioning, shadows, circuits.ee.IEC, calc}
\usepackage{caption}
\usepackage{subcaption}
\usepackage{wrapfig}
\usepackage{float}
\usepackage{eso-pic}
\usepackage[outline]{contour}
\usepackage[object=vectorian]{pgfornament} 

%----------------------------------------------------------------------------------------
% TABLES & LISTS
%----------------------------------------------------------------------------------------
\usepackage{booktabs}
\usepackage{threeparttable}
\usepackage{array}
\usepackage{longtable}
\usepackage{multirow}
\usepackage{titlesec}
\usepackage{enumitem}
\usepackage{lettrine}

%----------------------------------------------------------------------------------------
% CODE LISTINGS
%----------------------------------------------------------------------------------------
\usepackage{listings}
\lstset{
    basicstyle=\ttfamily\small\color{secondary},
    keywordstyle=\color{primary}\bfseries,
    stringstyle=\color{accent},
    commentstyle=\color{gray}\itshape,
    numberstyle=\tiny\color{gray},
    numbers=left,
    stepnumber=1,
    numbersep=8pt,
    showstringspaces=false,
    breaklines=true,
    frame=lines,
    backgroundcolor=\color{lightgray},
    rulecolor=\color{primary},
    captionpos=b,
    abovecaptionskip=10pt,
    literate={á}{{\'a}}1 {ã}{{\~a}}1 {é}{{\'e}}1 {ç}{{\c{c}}}1 {ó}{{\'o}}1 {õ}{{\~o}}1 {í}{{\'i}}1 {ú}{{\'u}}1 {Á}{{\'A}}1 {Ã}{{\~A}}1 {É}{{\'E}}1 {Ç}{{\c{C}}}1 {Ó}{{\'O}}1 {Õ}{{\~O}}1 {Í}{{\'I}}1 {Ú}{{\'U}}1
}

%----------------------------------------------------------------------------------------
% SPECIAL BOXES AND ENVIRONMENTS
%----------------------------------------------------------------------------------------
\usepackage{tcolorbox}
\tcbuselibrary{theorems, skins, breakable}

%----------------------------------------------------------------------------------------
% REFERENCES AND HYPERLINKS
%----------------------------------------------------------------------------------------
\usepackage[hidelinks, pdfencoding=auto, unicode]{hyperref}
\hypersetup{
    colorlinks=true,
    linkcolor=primary,
    filecolor=magenta,      
    urlcolor=engineering,
    citecolor=accent,
}
\usepackage{cleveref}

%----------------------------------------------------------------------------------------
% COLOR DEFINITIONS
%----------------------------------------------------------------------------------------
\definecolor{primary}{HTML}{1e3a8a}      
\definecolor{secondary}{HTML}{1f2937}    
\definecolor{accent}{HTML}{dc2626}       
\definecolor{lightgray}{HTML}{f8fafc}    
\definecolor{darkgray}{HTML}{374151}     
\definecolor{engineering}{HTML}{0369a1}  
\definecolor{lightblue}{HTML}{3b82f6}    

%----------------------------------------------------------------------------------------
% CUSTOM COMMANDS AND BOXES
%----------------------------------------------------------------------------------------
\contourlength{0.04em}

\newcommand*\splitdot{%
  \hspace{0.2em}\tikz[baseline=-0.3ex]{
    \fill[accent] (0,0) arc (0:180:0.25ex) -- cycle;
    \fill[lightblue] (0,0) arc (0:-180:0.25ex) -- cycle;
  }\hspace{0.2em}%
}

\newtcolorbox{futuristicbox}[1][]{
  enhanced, colback=secondary, colframe=primary!70, boxrule=0pt, arc=0pt,
  borderline west={1pt}{0pt}{primary!80},
  borderline east={1pt}{0pt}{accent!80},
  fontupper=\bfseries\color{white},
  halign=center, valign=center, boxsep=10pt, #1
}

\newtcolorbox{highlightbox}[1][]{ 
    enhanced, colback=lightgray, colframe=primary, boxrule=1pt, arc=5pt, 
    drop shadow={opacity=0.4, shadow xshift=2pt, shadow yshift=-2pt, fill=darkgray!80}, 
    breakable, coltitle=white, fonttitle=\bfseries\small\sffamily, colbacktitle=primary, 
    attach boxed title to top left={xshift=12pt, yshift=-5pt}, 
    boxed title style={arc=3pt, boxrule=0.5pt, drop shadow={opacity=0.4, shadow xshift=1pt, shadow yshift=-1pt, fill=darkgray}}, 
    left=10pt, right=10pt, top=12pt, bottom=8pt, #1 
}

\newcommand{\sectiondivider}{
    \begin{center}
        \begin{tikzpicture}
            \draw[primary!50, line width=1pt] (0,0) -- (5,0);
            \fill[accent] (2.5,0) circle (2pt);
        \end{tikzpicture}
    \end{center}
}

\newcommand{\PageBorder}{%
\AddToShipoutPictureBG*{%
  \begin{tikzpicture}[remember picture, overlay]
    \draw[secondary, line width=0.4pt] ([xshift=1.5cm,yshift=-1.5cm]current page.north west) rectangle ([xshift=-1.5cm,yshift=1.5cm]current page.south east);
    \draw[primary!80, line width=0.8pt] ([xshift=1.6cm,yshift=-1.6cm]current page.north west) rectangle ([xshift=-1.6cm,yshift=1.6cm]current page.south east);
  \end{tikzpicture}}%
}
\newcommand{\NoPageBorder}{\ClearShipoutPictureBG}
\newcommand{\RestorePageBorder}{\PageBorder}

\titleformat{\section}{\Large\bfseries\sffamily\color{primary}}{\thesection}{1em}{}[\vspace{-0.5em}\textcolor{accent}{\titlerule[0.8pt]}]
\titleformat{\subsection}{\large\bfseries\sffamily\color{secondary}}{\thesubsection}{1em}{}
\titleformat{\subsubsection}{\normalsize\bfseries\sffamily\color{darkgray}}{\thesubsubsection}{1em}{}

\setstretch{1.15} 
\setlength{\parindent}{1.2em} 
\setlength{\parskip}{0.5em}

%----------------------------------------------------------------------------------------
% METADATA
%----------------------------------------------------------------------------------------
\newcommand{\tituloTrabalho}{Benchmark de Clustering em NLP}
\newcommand{\subtituloTrabalho}{Avaliação de Embeddings e Algoritmos de Agrupamento}
\newcommand{\nomeAutor}{Cauã Vitor Figueredo Silva}
\newcommand{\matriculaAutor}{20220014216}
\newcommand{\emailAutor}{cauavitorfigueredo@gmail.com}
\newcommand{\cursoInstituicao}{Engenharia Elétrica – UFRN}
\newcommand{\disciplina}{Inteligência Artificial}

% Header & Footer
\pagestyle{fancy} 
\fancyhf{}
\fancyhead[LE,RO]{\small\color{primary}\textbf{\thepage}}
\fancyhead[RE]{\small\color{secondary}\textit{\nomeAutor}}
\fancyhead[LO]{\small\color{secondary}\textit{\tituloTrabalho}}
\fancyfoot[C]{\small\color{darkgray}\cursoInstituicao \ | 2025}
\renewcommand{\headrulewidth}{0.5pt} 
\renewcommand{\footrulewidth}{0.3pt}
\renewcommand{\headrule}{\hbox to\headwidth{\color{primary}\leaders\hrule height \headrulewidth\hfill}}
\renewcommand{\footrule}{\hbox to\headwidth{\color{primary}\leaders\hrule height \footrulewidth\hfill}}

%========================================================================================
% DOCUMENT START
%========================================================================================
\begin{document}

%----------------------------------------------------------------------------------------
% COVER PAGE
%----------------------------------------------------------------------------------------
\NoPageBorder
\begin{titlepage}
\thispagestyle{empty}
\begin{tikzpicture}[remember picture, overlay]
    \fill[secondary] (current page.north west) rectangle (current page.south east);
    \begin{scope}[opacity=0.1]
        \foreach \i in {1,...,20} {
            \pgfmathsetmacro{\x}{rand*10 + 10}
            \pgfmathsetmacro{\y}{rand*15}
            \fill[primary] (\x,\y) circle (0.15);
            \foreach \j in {1,...,3} {
                 \pgfmathsetmacro{\dx}{rand*3}
                 \pgfmathsetmacro{\dy}{rand*3}
                 \draw[lightblue, thin] (\x,\y) -- (\x+\dx, \y+\dy);
            }
        }
    \end{scope}
    \fill[primary] (current page.north west) -- ($(current page.north west) + (0,-1cm)$) -- ($(current page.north east) + (-10cm,-2.5cm)$) -- (current page.north east) -- cycle;
    \fill[accent] (current page.south west) -- ($(current page.south west) + (12cm, 2cm)$) -- ($(current page.south east) + (0, 1cm)$) -- (current page.south east) -- cycle;
\end{tikzpicture}

\begin{center}
    \vspace*{3cm}
    \vfill
    {\color{white}\large\bfseries\sffamily\textls[150]{\MakeUppercase{Universidade Federal do Rio Grande do Norte}}}\\[0.2cm]
    {\color{lightblue}\bfseries\sffamily\textls[100]{CENTRO DE TECNOLOGIA \splitdot DEPARTAMENTO DE ENGENHARIA ELÉTRICA}}\\[1.5cm]
    \renewcommand{\LettrineFontHook}{\color{accent}\bfseries}
    {\color{white}\fontsize{28}{32}\selectfont\bfseries\sffamily
    \tituloTrabalho
    }
    \vspace{0.5cm}
    \begin{futuristicbox}
    \sffamily\textls[100]{\MakeUppercase{\subtituloTrabalho}}
    \end{futuristicbox}
    \vfill
    {\color{white}\Large\bfseries\sffamily\textls[100]{\MakeUppercase{\nomeAutor}}}\\[0.3cm]
    {\color{white}\sffamily Matrícula: \texttt{\matriculaAutor} \splitdot \texttt{\emailAutor}}
    \vfill
    {\color{white}\large\bfseries\sffamily\textls[100]{NATAL - RN} \splitdot \textls[100]{\MakeUppercase{\today}}}
    \vspace{2cm}
\end{center}
\end{titlepage}

%----------------------------------------------------------------------------------------
% EXECUTIVE SUMMARY (RESUMO)
%----------------------------------------------------------------------------------------
\RestorePageBorder
\thispagestyle{plain}
\begin{center}
    {\Large\bfseries\color{primary}Resumo}
\end{center}

\lettrine[lines=3, lhang=0.1, loversize=0.2]{\color{accent}\textbf{E}}{ste trabalho} apresenta um estudo comparativo abrangente sobre a eficácia de diferentes modelos de embeddings e algoritmos de clustering aplicados a textos curtos em português (PT-6) e notícias em inglês (20 Newsgroups). O objetivo central foi identificar a melhor combinação técnica para agrupamento não supervisionado. Utilizou-se modelos de Transformer modernos (Sentence-BERT, GTE, BGE) e métodos clássicos (TF-IDF), avaliados por métricas externas (ARI, NMI, Pureza) e internas (Silhouette). Implementou-se um pipeline otimizado com redução de dimensionalidade (PCA) e paralelismo para viabilizar algoritmos densos como Spectral Clustering. Os resultados apontam que o modelo BGE (BAAI) combinado com K-Means ou GMM, reduzido a 100 dimensões, atinge desempenho de estado da arte (ARI > 0.94) no dataset de textos curtos, superando significativamente abordagens tradicionais.

\vspace{0.5cm}
\begin{center}
\begin{tikzpicture}
    \node[draw=primary, fill=lightgray, rounded corners=5pt, inner sep=8pt,
          drop shadow={opacity=0.4, shadow xshift=2pt, shadow yshift=-2pt, fill=darkgray!80}] {
    \begin{minipage}{0.8\textwidth}
        \centering
        \textbf{Palavras-chave:} NLP $\cdot$ Clustering $\cdot$ Embeddings $\cdot$ Transformers $\cdot$ Benchmark
    \end{minipage}
    };
\end{tikzpicture}
\end{center}
\newpage

%----------------------------------------------------------------------------------------
% MAIN CONTENT
%----------------------------------------------------------------------------------------
\pagestyle{fancy}
\pagenumbering{roman}
\tableofcontents
\newpage
\pagenumbering{arabic}
\setcounter{page}{1}

% ----------------------------------------------------------------------------
\section{Introdução}
O agrupamento de textos (text clustering) é uma tarefa fundamental em processamento de linguagem natural (NLP), permitindo a organização não supervisionada de grandes volumes de documentos.

\subsection{Contextualização e Problema}
Algoritmos de clustering clássicos enfrentam dificuldades com a alta dimensionalidade e esparsidade típicas de dados textuais. Modelos de linguagem baseados em Transformers (como BERT) mitigam isso gerando representações densas (embeddings), mas sua alta dimensionalidade (768+) introduz desafios computacionais ("Maldição da Dimensionalidade").

\subsection{Objetivos}
\begin{highlightbox}[title={Objetivo Geral}]
Avaliar e comparar a performance de diferentes embeddings (TF-IDF vs. Transformers) combinados com múltiplos algoritmos de clustering em datasets de características distintas.
\end{highlightbox}

Os objetivos específicos incluem:
\begin{itemize}
    \item Implementar pipeline robusto de pré-processamento, embedding e clustering.
    \item Otimizar a execução de algoritmos complexos (Spectral, DBSCAN) via PCA.
    \item Comparar métricas de qualidade (ARI, NMI, Silhouette).
\end{itemize}

\sectiondivider

% ----------------------------------------------------------------------------
\section{Metodologia}

\subsection{Datasets}
\begin{itemize}
    \item \textbf{20 Newsgroups (20NG-6):} Subconjunto de 6 classes (comp, rec, sci, etc.), $\sim$6000 documentos. Textos longos, vocabulário técnico.
    \item \textbf{PT-6 (Textos Curtos):} Dataset sintético/coletado com 6 classes em português. Textos curtos, desafiador pela pouca informação semântica.
\end{itemize}

\subsection{Modelos de Embedding}
Foram avaliados quatro métodos:
\begin{enumerate}
    \item \textbf{TF-IDF + SVD (Baseline):} Estatístico, baseado em frequência de termos.
    \item \textbf{Sentence-BERT (SBERT):} Modelo siamês clássico (`paraphrase-multilingual`).
    \item \textbf{GTE-Base:} General Text Embedding, otimizado para recuperação.
    \item \textbf{BGE-M3:} Modelo da BAAI, estado da arte em representação densa multilingue.
\end{enumerate}

\subsection{Algoritmos e Otimização}
Algoritmos utilizados: \textbf{K-Means, GMM, Agglomerative, DBSCAN, Spectral, HDBSCAN}.

\begin{highlightbox}[title={Otimização Crítica}]
Para viabilizar o \textit{Spectral Clustering} e \textit{DBSCAN} em vetores densos, aplicou-se \textbf{PCA (Principal Component Analysis)}, reduzindo a dimensionalidade de 768/1024 para 100 componentes, preservando $>$95\% da variância explicada. O Spectral Clustering foi configurado para usar \texttt{affinity='nearest\_neighbors'} (grafo esparso) ao invés de RBF.
\end{highlightbox}

\sectiondivider

% ----------------------------------------------------------------------------
\section{Resultados e Discussão}

Os experimentos foram rodados em todas as 48 combinações possíveis (2 datasets $\times$ 4 embeddings $\times$ 6 algoritmos).

\subsection{Análise Quantitativa}
A Tabela \ref{tab:resumo_pt6} apresenta os melhores resultados obtidos para o dataset PT-6.

\begin{table}[H]
\centering
\caption{Melhores Resultados no Dataset PT-6}
\label{tab:resumo_pt6}
\begin{tabular}{@{}lccc@{}}
\toprule
\textbf{Embedding} & \textbf{Algoritmo} & \textbf{ARI} & \textbf{NMI} \\ \midrule
BGE-M3 & K-Means & \textbf{0.941} & \textbf{0.935} \\
BGE-M3 & GMM & 0.940 & 0.934 \\
SBERT & Spectral & 0.816 & 0.847 \\
TF-IDF & K-Means & 0.450 & 0.500 \\ \bottomrule
\end{tabular}
\end{table}

\subsection{Visualização dos Clusters}
A Figura \ref{fig:cluster_vis} demonstra a projeção UMAP dos clusters gerados pelo melhor modelo no dataset PT-6.

\begin{figure}[H]
    \centering
    \includegraphics[width=0.85\textwidth]{../results/figures/compare_pt6_bge_umap.png}
    \caption{Comparação: Classes Reais (Esq) vs. Clusters BGE + K-Means (Dir).}
    \label{fig:cluster_vis}
\end{figure}

Observa-se na Figura \ref{fig:cluster_vis} que o BGE agrupou os dados quase perfeitamente (ARI 0.94), com separação nítida visualmente, validando as métricas quantitativas.

\sectiondivider

% ----------------------------------------------------------------------------
\section{Conclusão}
Este trabalho demonstrou que o uso de embeddings densos modernos (especialmente BGE-M3) combinado com redução de dimensionalidade (PCA) supera largamente abordagens tradicionais baseadas em frequência (TF-IDF). A otimização implementada permitiu executar algoritmos computacionalmente custosos em tempo hábil. O pipeline desenvolvido é robusto e pode ser aplicado a outros domínios de NLP.

\begin{thebibliography}{9}
\bibitem{reimers2019} Reimers, N., \& Gurevych, I. (2019). Sentence-BERT: Sentence Embeddings using Siamese BERT-Networks. EMNLP.
\bibitem{xiao2023} Xiao, S. et al. (2023). C-Pack: Packaged Resources To Advance General Chinese Embedding. BAAI.
\end{thebibliography}

\end{document}
